%\begin{figure*}[!h]
\begin{subfigure}{0.26\linewidth}
\centering
\includegraphics[width=0.9\linewidth]{Images/Tornado/zls.pdf}
\vspace{-2mm}
\caption{$ZLS_{T}$}
\label{fig:tornado_zls}
\end{subfigure}
\begin{subfigure}{0.26\linewidth}
\centering
\includegraphics[width=0.9\linewidth]{Images/Tornado/fls_05.pdf}
\vspace{-2mm}
\caption{$ZLS_{T}$ + $FLS_{T,0.5}$}
\label{fig:tornado_fls}
\end{subfigure}
\begin{subfigure}{0.26\linewidth}
\centering
\includegraphics[width=0.9\linewidth]{Images/Tornado/fcls_68.pdf}
\vspace{-2mm}
\caption{$ZLS_{T}$ + $FCLS_{T,68\%}$}
\label{fig:tornado_fcls}
\end{subfigure}
\hfill
\begin{subfigure}{0.17\linewidth}
\centering
\includegraphics[width=\linewidth]{Images/Tornado/scatterplot.pdf}
\vspace{-4mm}
\caption{Attribute space 2D scatterplot and trait (rectangular selection).} 
\label{fig:tornado_scatterplot}
\end{subfigure}
\vspace{-3mm}
\caption{Visualization of EF-5 tornado vortices using vorticity magnitude and pressure pertubation attributes.}
\vspace{-2mm}
\label{fig:tornado}
\end{figure*}

\begin{figure*}[!h]
\begin{subfigure}{0.24\linewidth}
\centering
\includegraphics[width=0.95\linewidth]{Images/RedSeaEddy/scatterplot.pdf}
\vspace{-1mm}
\caption{2D scatterplot of $\mathcal{A}$ and traits. We use $T = \left\{T_{A}, T_{B}\right\}$.} 
\label{fig:rse_scatterplot}
\end{subfigure}
\hfill
\begin{subfigure}{0.245\linewidth}
\centering
\includegraphics[width=\linewidth]{Images/RedSeaEddy/zls.pdf}
\vspace{-5mm}
\caption{$ZLS_{T}$}
\label{fig:rse_zls}
\end{subfigure}
\begin{subfigure}{0.245\linewidth}
\centering
\includegraphics[width=\linewidth]{Images/RedSeaEddy/fcls_50.pdf}
\vspace{-5mm}
\caption{+ $FCLS_{T,50\%}$}
\label{fig:rse_fls}
\end{subfigure}
\begin{subfigure}{0.245\linewidth}
\centering
\includegraphics[width=\linewidth]{Images/RedSeaEddy/fcls_68.pdf}
\vspace{-5mm}
\caption{+ $FCLS_{T,68\%}$}
\label{fig:rse_fcls}
\end{subfigure}
\vspace{-2mm}
\caption{Visualization of anticyclonic~($T_{A}$, red) and cyclonic~($T_{B}$, blue) eddies in the Gulf of Aden and part of the Red Sea using the derived attributes of vorticity magnitude and the z-component of curl. For this ensemble data set~\cite{sanikommu2020impact}, the uncertainty resulted in $FCLS_{T,C}$ visualizing additional tracks and regions with eddies. The orange boxes in (b) and (c) highlight one such example.}
%\vspace{-2mm}
\label{fig:rse}
\end{figure*}

\begin{figure*}[!h]
\begin{subfigure}{0.18\linewidth}
\centering
\includegraphics[width=\linewidth]{Images/EthaneDiol/scatterplot.pdf}
\caption{2D scatterplot of $\mathcal{A}$ and traits. We use $T = \left\{T_{A}, T_{B}, T_{C}, T_{D}\right\}$.} 
\label{fig:ethanediol_scatterplot}
\end{subfigure}
\begin{subfigure}{0.16\linewidth}
\centering
\includegraphics[width=\linewidth]{Images/EthaneDiol/zls.pdf}
\caption{$ZLS_{T}$}
\label{fig:ethanediol_zls}
\end{subfigure}
\begin{subfigure}{0.16\linewidth}
\centering
\includegraphics[width=\linewidth]{Images/EthaneDiol/fcls_cb_68.pdf}
\caption{+ $FCLS_{T_{A},68\%}$}
\label{fig:ethanediol_fcls_cb}
\end{subfigure}
\begin{subfigure}{0.16\linewidth}
\centering
\includegraphics[width=\linewidth]{Images/EthaneDiol/fcls_ncb_68.pdf}
\caption{+ $FCLS_{T_{B},68\%}$}
\label{fig:ethanediol_fcls_ncb}
\end{subfigure}
\begin{subfigure}{0.16\linewidth}
\centering
\includegraphics[width=\linewidth]{Images/EthaneDiol/fcls_oa_68.pdf}
\caption{+ $FCLS_{T_{C},68\%}$}
\label{fig:ethanediol_fcls_oa}
\end{subfigure}
\begin{subfigure}{0.16\linewidth}
\centering
\includegraphics[width=\linewidth]{Images/EthaneDiol/fcls_ca_68.pdf}
\caption{+ $FCLS_{T_{D},68\%}$}
\label{fig:ethanediol_fcls_ca}
\end{subfigure}
\caption{The covalent bonds~($T_{A}$, blue), non-covalent bond~($T_{B}$, green), oxygen atoms~($T_{C}$, red), and carbon atoms~($T_{D}$, yellow) of an ethanediol molecule are visualized using the electron density~(Rho) and reduced gradient~(s) attributes. These attributes are related exponentially in regions where no chemical interaction occurs and we selected our traits accordingly. In this case, we found $FCLS_{T,C}$ collectively visualized elements of the topological structure of the molecule.}
\label{fig:ethanediol}
\end{figure*}

%\begin{figure*}[!h]
\begin{subfigure}{0.195\linewidth}
\centering
\includegraphics[width=\linewidth]{Images/Mantel/zls.pdf}
\vspace{-5mm}
\caption{$ZLS_{T}$}
\label{fig:mantel_zls}
\end{subfigure}
\begin{subfigure}{0.195\linewidth}
\centering
\includegraphics[width=\linewidth]{Images/Mantel/fcls_68.pdf}
\vspace{-5mm}
\caption{$ZLS_{T}$ + $FCLS_{T,68\%}$}
\label{fig:mantel_fls}
\end{subfigure}
\begin{subfigure}{0.29\linewidth}
\centering
\includegraphics[width=0.9\linewidth]{Images/Mantel/fcls_68_v2.pdf}
\caption{View of $ZLS_{T}$ + $FCLS_{T,68\%}$ revealing the uncertain structure and spatial proximity of features identified by the selected traits.}
\label{fig:mantel_fcls}
\end{subfigure}
\hfill
\begin{subfigure}{0.295\linewidth}
\centering
\includegraphics[width=\linewidth]{Images/Mantel/scatterplot.pdf}
%\vspace{-2mm}
\caption{2D scatterplot of $\mathcal{A}$ and traits. For the traits, we used extremes of temperature anomaly and negative spin-transition-induced density anomaly~(stida) to visualize flow patterns. We use $T = \left\{T_{A}, T_{B}\right\}$.} 
\label{fig:mantel_scatterplot}
\end{subfigure}
\vspace{-2mm}
%\caption{Visualization of a subset of the spatial domain for the mantel data using temperature anomaly and spin-transition-induced density anomaly~(stida). We cosnidered extremes of temperature anomaly and negative spin-transition-induced density anomaly to visualize the flow patterns of rising hot plumes~($T_{A}$, red) and sinking material~($T_{B}$, blue) in the data.}
\caption{Visualization of rising hot plumes~($T_{A}$, red) and sinking material~($T_{B}$, blue) flow patterns in the Earth's mantel convection data using the temperature anomaly and spin-transition-induced density anomaly attributes. We considered a rectlinearly sampled mesh for a subset of the spatial domain.}
\label{fig:mantel}
\end{figure*}

%\begin{figure*}[!ht]
%\begin{subfigure}{0.195\linewidth}
%\centering
%\includegraphics[width=0.85\linewidth]{Images/Tangle/gt.pdf}
%\vspace{-2mm}
%\caption{Ground truth, $isoval=62$}
%\label{fig:tangle_gt}
%\end{subfigure}
\begin{subfigure}{0.19\linewidth}
\centering
\includegraphics[width=0.9\linewidth]{Images/Tangle/zls.pdf}
\vspace{-2mm}
\caption{$ZLS_{T}$}
\label{fig:tangle_zls}
\end{subfigure}
\begin{subfigure}{0.19\linewidth}
\centering
\includegraphics[width=0.9\linewidth]{Images/Tangle/fcls_50.pdf}
\vspace{-2mm}
\caption{+ $FCLS_{T,50\%}$}
\label{fig:tangle_fcls_50}
\end{subfigure}
\begin{subfigure}{0.19\linewidth}
\centering
\includegraphics[width=0.9\linewidth]{Images/Tangle/fcls_68.pdf}
\vspace{-2mm}
\caption{+ $FCLS_{T,68\%}$}
\label{fig:tangle_fcls_68}
\end{subfigure}
\begin{subfigure}{0.19\linewidth}
\centering
\includegraphics[width=0.9\linewidth]{Images/Tangle/fcls_95.pdf}
\vspace{-2mm}
\caption{+ $FCLS_{T,95\%}$}
\label{fig:tangle_fcls_95}
\end{subfigure}
\begin{subfigure}{0.225\linewidth}
\centering
\includegraphics[width=\linewidth, trim={0.2cm 0cm 0cm 0cm}, clip]{Images/Tangle/comparison.pdf}
\vspace{-5mm}
\caption{Comparison of $FCLS_{T,C}$}
\label{fig:tangle_gt}
\end{subfigure}
\vspace{-2mm}
\caption{Visualization of the analytical tangle function with a focus on uncertainty in linking regions between multiple blobs. We used $T=[0,62]$. We use the ``+'' symbol to indicate augmentation to $ZLS_{T}$. As expected for this data set, we found $FCLS_{T,C}$~(visualized as 25\% opacity level-sets) are visible in the linking regions and form wider envelopes as the confidence interval increases from 50\%~(c) to 95\%~(e).} 
%\caption{Visualization of sensitivity of the tangle function near values that form links between the multiple blobs. We use $T=[0,62]$.\fix{start a story}}
\label{fig:tangle}
\end{figure*}

We demonstrated the use of feature confidence level-sets using five data sets.
%
Specifically, we considered an analytical tangle function~\cite{knoll2009fast}, an EF-5 Tornado~\cite{atmos10100578}, an ethanediol molecule from a chemistry simulation, Red Sea and Gulf of Aden~(RSGOA) eddy ensemble~\cite{sanikommu2020impact}, and Earth's mantel convection~\cite{shahnas2017mid} data (see additional material).
%
We defined between one to four traits per data set based on features of interest. 
%
%While our study demonstrated feature confidence level-sets for univariate and bivariate data, the approach can be applied to higher dimensions.\fix{Can we get an example in for a third?***}
%\fix{(Should we remove the part "the approach can be applied to higher dimensions" since we do not have results for n>2, but we can state the same in the future work section?)}.
%
In this study, each attribute was represented using a ${\mu}$ and ${\sigma}$ field. 
%
For the RSGOA data set, we computed ${\mu}$ and ${\sigma}$ fields using 20 ensemble members. 
%
For other data sets, we synthetically estimated ${\sigma}$ for each scalar field of the multivariate data at each grid point by sampling the local neighborhood.
%
To evaluate our technique, we visualized the $ZLS_{T}$ both in isolation and augmented with $FCLS_{T,C}$. 
%
When visualized together, the $ZLS_{T}$ is shown using an opaque level-set, and the $FCLS_{T,C}$ is shown using a level-set colored with the same hue and 25\% opacity.
%
We used VisIt~\cite{childs2012visit} to extract and render smooth level-sets using the pseudocolor plot and isosurface operator.

%also augment it with a semi-opaque, single feature level-set $FLS_{T,K}$ for some level $K$, or a feature confidence level-set $FCLS_{T,C}$ for some confidence $C$ (and level $\epsilon$).
%

%For all data sets considered, as expected we found, $FLS_{T,K}$ has the shortcoming of discernibility.
%
%The $FLS_{T,K}$ level-set typically formed a ``shell'' like structure.
%
Across all data sets, the shape of $FCLS_{T,C}$ corresponded to the uncertainty of the data in the spatial domain.
%
For example, for the analytical tangle function where uncertainty is higher near the links between the blobs for the trait specified, we found, comparing Figures~\ref{fig:tangle_fcls_68} and~\ref{fig:tangle_fcls_95}, the $FCLS_{T,C}$ envelope expanded between the links in response to increasing the value of $C$, but not significantly on the exterior of the blob surface.
%
For the Tornado data set, we specify a trait using three attributes related to vorticity, including negative pressure pertubation~(prespert) values that are associated with the updraft rotational mechanics of an evolving tornado. 
%
$FCLS_{T,C}$ visualize weaker vortices in proximity to the primary vortex in Figure~\ref{fig:tornado_fcls}.
%
Such visualizations could be useful in visualizing vortex merges during the formation of a multiple vortex tornado~\cite{atmos10100578}. 
%

For the Red Sea eddy simulation data, we visualized anticyclonic~(red isosurfaces) and cyclonic~(blue isosurfaces) eddies in Figure~\ref{fig:rse}. 
%
Using the $\mu$ field to compute $ZLS_{T}$ for the two specified traits~(Figure~\ref{fig:rse_scatterplot}) reveals regions where large eddies in the Gulf of Aden and eddy tracks in the Red Sea exist, as well as the type of eddy in Figure~\ref{fig:rse_zls}.
%
To investigate the uncertainty of outcomes across ensemble members, the $\mu$ and $\sigma$ fields are utilized to compute $FCLS_{T,C}$ for 50\% and 68\% confidence intervals.
%
Besides showing larger regions of eddys in the Gulf of Aden, $FCLS_{T,C}$ visualizes the possible existence of additional eddy tracks in the Red Sea across ensemble members for the specific trait selection, which is not seen in the $ZLS_{T}$ derived from the mean fields.
%, with increased connectivity as the confidence interval widens.
%
Figures~\ref{fig:rse_fls} and~\ref{fig:rse_fcls} are annotated to highlight one such example.
%

%
In the ethanediol data set, electron density and reduced gradient are related exponentially in regions where no chemical interactions occur~(main separating axis of the scatterplot in Figure~\ref{fig:ethanediol_scatterplot}).
%
%In Figure~\ref{fig:ethanediol_scatterplot}, the main separating axis of the scatterplot plot forms this region of no chemical interaction. 
%
Our trait selections in attribute space are off this axis and correspond to regions with significant chemical interactions.
%
%Figure~\ref{fig:ethanediol_zls} visualizes features extracted using four traits.
%
In this case, we found $FCLS_{T,C}$ of individual traits visualized the boundaries of non-chemical interactivity for each feature. 
%
Figures~\ref{fig:ethanediol_fcls_cb} and~\ref{fig:ethanediol_fcls_ncb} show $FCLS_{T,C}$ for the covalent and non-covalent bond form enclosing structures primarily around the respective features.
%
Similarly, in Figures~\ref{fig:ethanediol_fcls_oa} and~\ref{fig:ethanediol_fcls_ca}, $FCLS_{T,C}$ of each trait are observed in regions of influence of each atom, conveying the proximity of the traits in attribute space and the uncertainty in the data.
%
Figure~\ref{fig:ethanediol_fcls_oa} contains occluded $FCLS_{T_{C},C}$ on the inside of each carbon atom~(yellow). 
%For the covalent bond and non-covalent bonds T_{B}$ and $T_{C}$, we found enclosing level-sets, while for $T_{A}$ and $T_{D}$.
%
Overall, by leveraging the information pertaining to field distribution~(${\mu}$, ${\sigma}$), $FCLS_{T,C}$ provided secondary structure visualization based on uncertainty.
%
%The feature confidence level-sets for each feature
%, topological structure information for the ethanediol molecule~(Figure~\ref{fig:ethanediol}), regions with anticyclonic and cyclonic eddies across ensemble members~(Figure~\ref{fig:rse}), and the proximity as well as relationship of contrasting features in the spatial domain for the mantel data set~(Figure~\ref{fig:mantel_fcls_68_v2}).
%


%Importantly, for the traits selected, the secondary structures produced using $FCLS_{T,C}$ did not overlap one another.
%
%In contrast, for the same traits, $FLS_{T,K}$ resulted in overlapping and occluding level-sets~(see Figure~\ref{fig:rse_fls}).
%
%We provide supplementary visualizations and comparisons in the additional material. 
 
