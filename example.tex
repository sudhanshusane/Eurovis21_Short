\begin{figure}[!b]
\centering
\includegraphics[width=\linewidth, trim={1cm 0cm 0.5cm 0cm}, clip]{Images/example_narrow.pdf}
%\vspace{-5mm}
\caption{A notional example showing the steps involved in generating the ``zero'' feature level-set $ZLS_{T}$~(top row) and feature confidence level-set $FCLS_{T,C}$~(bottom row) for an uncertain univariate field represented using $mean$~(a) and $SD$~(e). 
%
For this example, we use trait $T=[2.5, 3.5]$ and confidence $C=68\%$, i.e., $Z=1$.
%
Assuming a unit distance between adjacent grid points,
%
$bvolume_{T,C}$~(g) and $distance_{T,C}$ (not shown) would be equivalent for this example.
}
%\vspace{-2mm}
\label{fig:example}
\end{figure}

% \fix{The next two sentences could be removed for brevity if we state "top row" and "bottom row" in the first sentence. Your call.} The top row~(b-e) shows the steps to compute $FLS_{T}$. The bottom row~(g-j) shows the steps to compute $FCLS_{T}$. 
%$ZLS_{T}$ represents the ``zero'' level-set extracted using $distance_{T}$.}
% In comparison to using $FLS_{T}$, for uncertain multivariate data, by leveraging the information pertaining to field distribution ($mean$, $SD$), $FCLS_{T,C}$ can provide a confidence controlled visualization of uncertain regions and improve secondary structure visualization for a specific trait.}

