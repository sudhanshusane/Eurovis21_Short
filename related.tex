%Feature confidence level-sets target scientific uncertain multivariate spatial data.
%%
Although there have been efforts to visualize non-spatial uncertain multivariate data, they are limited for spatial data.
%
For comprehensive overviews, we refer readers to state-of-the-art reports for uncertainty visualization~\cite{Bonneau2014} \fix{(Here we may add ~\cite{JohnsonSanderson2003, PotterRosenJohnson2012})} and multivariate spatial data visualization~\cite{he2019multivariate}.
%
In this section, we restrict our discussion to works most relevant to this study.

In the recent past, there are two notable multivariate spatial data visualization efforts: fiber surfaces and feature level-sets.
%
Fiber surfaces proposed by Carr \textit{et al.}~\cite{carr2015fiber} are the generalization of isosurfaces to bivariate data and involve modifying the Marching Cube's algorithm.
%
Parallelized implementations~\cite{klacansky2016fast}, direct volume rendering using higher-order interpolation schemes~\cite{wu2016direct}, and extensions to multivariate data~\cite{blecha2020fiber} have been studied for fiber surfaces.  
%
Feature level-sets, as previously mentioned, are the generalization of isosurfaces to multivariate data and were proposed by Jankowai and Hotz~\cite{jankowai2020feature}.
%
Smoothing of the distance field using Guassian kernels~\cite{nguyen2020visualization}, application to tensor data~\cite{jankowai2020tensor}, and use within visualization frameworks~\cite{jonsson2020inviwo} have been studied for feature level-sets.
%
%In~\cite{jankowai2020feature}, the authors identify multiple opportunities for future research including the definition of the distance metric, addressing discernibility shortcomings, performance optimizations, trait specification interfaces, etc.
%
Current literature, however, lacks research proposing techniques to visualize uncertain multivariate data.
%
%In this paper, we propose a technique to visualize such data via feature level-sets.


%In the recent past, there are two notable multivariate spatial data visualization efforts: fiber surfaces and feature level-sets.
%%
%Fiber surfaces proposed by Carr et al.~\cite{carr2015fiber} are the generalization of isosurfaces to bivariate data and involve modifying the Marching Cube's algorithm.
%%
%Klacansky et al.~\cite{klacansky2016fast} proposed a parallelized implementation to speedup generation of exact fiber surfaces.
%%
%Wu et al.~\cite{wu2016direct} perform accurate direct volume rendering of fiber surfaces using higher-order interpolation schemes.
%%
%More recently, fiber surfaces, have been extended to multivariate data by Blecha et al.~\cite{blecha2020fiber}.
%%
%
%Feature level-sets, as previously mentioned, are the generalization of isosurfaces to multivariate data and were proposed by Jankowai and Hotz~\cite{jankowai2020feature}.
%%
%In ~\cite{jankowai2020feature}, the authors identify multiple opportunities for future research including the definition of the distance metric, performance optimizations, trait specification interfaces, etc.
%%
%Nguyen et al.~\cite{nguyen2020visualization} proposed first smoothing of the distance field using Gaussian kernels before extracting level-sets to visualize coherent structures in Taylor-Couette turbulence flows.
%%
%Jankowai et al.~\cite{jankowai2020tensor} explored its application for tensor data. 
%%
%Further, the feature-level sets technique has been implemented for multivariate data visualization within the Inviwo visualization framework~\cite{jonsson2020inviwo}. 
%%
%In this paper, we extend and investigate the feature-level sets technique for uncertain multivariate data. 

%There are several research works aiming to address uncertainty in scientific visualization.
%
%However, faced with occlusion challenges these techniques have largely focused on univariate data~\cite{}.
%
%In this work, 
\fix{Several recent research investigated quantification and visualization of uncertainty in univariate isosurfaces~\cite{CRJ:Gri2004, pothkow:2011:PMC, AthawaleJohnson2019}.} We extend the method by Zehner \textit{et al.}~\cite{zehner2010visualization} to visualize uncertainty using confidence isosurfaces. \fix{(or We extend the idea of confidence isosurfaces proposed by Zehner \textit{et al.}~\cite{zehner2010visualization} for uncertainty visualization of univariate data to multivariate data.)}
%
In addition to offering straightforward generalization to multivariate data, confidence isosurfaces are determined on the basis of a specific confidence interval percentage and thus, provides an intuitive understanding by producing different shapes of isosurfaces due to uncertainty. 
%
%In contrast, correctly identifying isovalues to render multiple level-sets often involves trial and error as well as spatial extents knowledge. 
%Feature level-sets are computed based on a trait specified in attribute space.
%%
%The trait is analogous to an isovalue for the univariate case, except that for a non-empty feature level-set the trait must exist in attribute space.
%%
%Traditionally, for a specific trait, first a binary volume is computed, followed by a Euclidean distance transform to generate a distance field, and lastly, isosurfacing of the distance field.
%%
%At a distance of zero (or a small threshold), the feature in the spatial domain as defined by the trait in attribute space can be extracted.
%%
%Additional level-sets can be extracted from the distance field to provide insight regarding the feature structure.
%%
%Figure~\ref{} illustrates the visualization pipeline for the generation of feature-level sets.
%
%%
%In this paper, we propose replacing the distance field level-sets with level-sets showing uncertainty. 
%%
%This approach in effect improves the performance of the visualization pipeline by eliminating the Euclidean distance transformation.
%%
%Instead, we compute binary volumes for the feature level-set and the feature confidence level-set. 
%%
%Zehner et al.~\cite{zehner2010visualization} proposed the use of isosurface envelopes to show confidence for the univariate case.
%%
%Using the same definition and extending it to the multivariate case, a feature confidence level-set indicates within which volume the feature will lie with a certain confidence.
%%
%We illustrate a notional example of a feature confidence level-set for bivariate data in Figure~\ref{}.
