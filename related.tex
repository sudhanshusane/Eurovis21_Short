%Feature confidence level-sets target scientific uncertain multivariate spatial data.
%%
%Although there have been efforts to visualize non-spatial uncertain multivariate data, they are limited for spatial data.
%
For comprehensive overviews, we refer readers to reports for uncertainty visualization~\cite{Bonneau2014,johnson2003next,potter2011quantification} and multivariate spatial data visualization~\cite{he2019multivariate}.
%
In this section, we restrict our discussion to works most relevant to this study.

Two notable multivariate spatial data visualization efforts of the recent past are fiber surfaces and feature level-sets.
%
Fiber surfaces proposed by Carr et al.~\cite{carr2015fiber}, are the generalization of isosurfaces to bivariate data and involve modifying the Marching Cube's algorithm.
%
Parallelized implementations~\cite{klacansky2016fast}, direct volume rendering using higher-order interpolation schemes~\cite{wu2016direct}, and extensions to multivariate data~\cite{blecha2020fiber} have been studied for fiber surfaces.  
%
Feature level-sets, as previously mentioned, are the generalization of isosurfaces to multivariate data and were proposed by Jankowai and Hotz~\cite{jankowai2020feature}.
%
Further studies of feature level-sets have focused on adapting the distance metric and smoothing of the distance field using Guassian kernels~\cite{nguyen2020visualization}, application to tensor data~\cite{jankowai2020tensor}, and use within visualization frameworks~\cite{jonsson2020inviwo}.
%
%In~\cite{jankowai2020feature}, the authors identify multiple opportunities for future research including the definition of the distance metric, addressing discernibility shortcomings, performance optimizations, trait specification interfaces, etc.
%
Another recent work, Hazarika et al.~\cite{hazarika2018codda} first performed lossy in situ reduction to summarize data via copula-based distribution models. 
%
Next, in response to bivariate data analysis queries, they visualized probability fields generated by sampling the stored data summary. 
%

%The current literature, however, lacks research proposing techniques to visualize uncertain multivariate data.
%
%In this paper, we propose a technique to visualize such data via feature level-sets.


%In the recent past, there are two notable multivariate spatial data visualization efforts: fiber surfaces and feature level-sets.
%%
%Fiber surfaces proposed by Carr et al.~\cite{carr2015fiber} are the generalization of isosurfaces to bivariate data and involve modifying the Marching Cube's algorithm.
%%
%Klacansky et al.~\cite{klacansky2016fast} proposed a parallelized implementation to speedup generation of exact fiber surfaces.
%%
%Wu et al.~\cite{wu2016direct} perform accurate direct volume rendering of fiber surfaces using higher-order interpolation schemes.
%%
%More recently, fiber surfaces, have been extended to multivariate data by Blecha et al.~\cite{blecha2020fiber}.
%%
%
%Feature level-sets, as previously mentioned, are the generalization of isosurfaces to multivariate data and were proposed by Jankowai and Hotz~\cite{jankowai2020feature}.
%%
%In ~\cite{jankowai2020feature}, the authors identify multiple opportunities for future research including the definition of the distance metric, performance optimizations, trait specification interfaces, etc.
%%
%Nguyen et al.~\cite{nguyen2020visualization} proposed first smoothing of the distance field using Gaussian kernels before extracting level-sets to visualize coherent structures in Taylor-Couette turbulence flows.
%%
%Jankowai et al.~\cite{jankowai2020tensor} explored its application for tensor data. 
%%
%Further, the feature-level sets technique has been implemented for multivariate data visualization within the Inviwo visualization framework~\cite{jonsson2020inviwo}. 
%%
%In this paper, we extend and investigate the feature-level sets technique for uncertain multivariate data. 

%There are several research works aiming to address uncertainty in scientific visualization.
%
%However, faced with occlusion challenges these techniques have largely focused on univariate data~\cite{}.
%
%In this work, 
Several research works have investigated quantification and visualization of uncertainty in univariate isosurfaces~\cite{CRJ:Gri2004, pothkow:2011:PMC, AthawaleJohnson2019}.
%
In our work, we extend the concept of confidence isosurfaces for uncertainty visualization of univariate data proposed by Zehner et al.~\cite{zehner2010visualization} to multivariate data. 
%
%In addition to offering straightforward generalization to multivariate data, 
Confidence isosurfaces are determined on the basis of a specific confidence interval percentage and thus, provide an intuitive understanding by producing different shapes of isosurfaces due to uncertainty. 
%
%Overall, we contribute a technique for uncertainty visualization for multivariate data based on the standard deviation of distributions and feature level-sets.
