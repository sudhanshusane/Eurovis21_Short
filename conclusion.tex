In this paper, we proposed feature confidence level-sets and demonstrated their use for uncertain multivariate data visualization.
%
Several opportunities, however, remain for future work in this direction.
%There remain, however, several opportunities for future work in this direction.
%
Similar to feature level-sets~\cite{jankowai2020feature}, addressing discernibility and intuitive trait specification interfaces for high-dimensional data with uncertainty are challenges for feature confidence level-sets.
%
Considering the impact of the source of uncertainty and representation of the multivariate data, we plan to investigate the use of feature confidence level-sets on scientific data from lossy compressors such as ZFP~\cite{lindstrom2014fixed}, as well as parametric and non-parametric density models. 
%
Further, we aim to pursue visualization of interquartile ranges for uncertain multivariate data and performance optimizations that can be introduced to render implicit feature and feature confidence level-sets.
%

%These include evaluations of feature confidence level-sets on
%parametric and non-parametric density models, application of attribute-specific confidence interval percentages, 
%visualization of derived feature probability fields, and visualizations of mappings of uncertain multivariate data between the spatial domain and attribute space.
%
%Further, although we considered a simplified trait definition, intuitive trait specification interfaces 
%that include controls to specify confidence for higher dimensional data remain an open research challenge. 

Overall, we contributed a technique to visualize uncertain multivariate data based on confidence isosurfaces and feature level-sets.
%
Our study demonstrated the ability of the approach to visualize regions of uncertainty in relation to a specific trait or feature, and visualize secondary feature structures based on uncertainty.
