%Uncertain multivariate data are produced by most scientific simulations and pose significant challenges to visualize effectively.
%
Recent advancements in multivariate data visualization have opened new research opportunities for the visualization community.
%
In this paper, we propose 
%and qualitatively evaluate 
an uncertain multivariate data visualization technique called \textit{feature confidence level-sets}.
%
Our technique extends the concept of confidence isosurfaces for univariate data to \textit{feature level-sets}, i.e., the generalization of isosurfaces to multivariate data.
%
Feature confidence level-sets are computed by considering the trait for a specific feature, a confidence interval, and the distribution of data at each grid point in the domain.
%
%For a specific trait definition and confidence interval, we compare the use of multiple feature level-sets to a single feature level-set augmented with a feature confidence level-set.
%
Using uncertain multivariate data sets, we demonstrate the utility of the technique to visualize regions with uncertainty in relation to the specific trait or feature, and the ability of the technique to provide improved secondary feature structure visualization.


\fix{Would it be a good idea to break the third sentence (The words "extends" and "generalization" in the same sentence are making sentence difficult to comprehend in my view)? something like this maybe?: Conceptually, feature level-sets refer to level-sets of multivariate data. Our proposed feature confidence level-sets extend the existing idea of univariate confidence isosurfaces to multivariate feature level-sets . }
%of feature structure. % in the domain space for a specific trait defined in attribute space.
%For multiple uncertain multivariate data sets, we demonstrate that in addition to visualizing regions of uncertainty, our approach provides improved insight of feature structure than multiple feature level-sets. 
%For multiple uncertain multivariate data sets, we demonstrate our approach provides improved insight of feature structure and regions with uncertainty than multiple feature level-sets.
%
%To visualize uncertainty, we augmented the mean zero level-set generated for a trait with semi-opaque confidence interval level-sets.
%%
%We find that for uncertain data, our approach provides greater insight than visualization of Euclidean distance field level-sets typically used with the zero level-set.
%%
%To improve performance, our technique accelerates the visualization pipeline by avoiding the computation of a Euclidean distance field. 
%%
%Instead, we perform fast parallel computations of multiple binary volumes that satisfy trait and confidence interval criteria, followed by direct isosurfacing of the binary volumes.
%%
%We demonstrate the effectiveness of the technique using five uncertain multivariate data sets including analytical data, ensemble data, and real-world data with synthetic noise.
